
\begin{figure}[ptb]
\includegraphics[scale=0.30]{logo.jpg}\hspace{6cm}
\includegraphics[scale=0.90]{logo_dc.jpg}
\end{figure}

%Datos de la caratula
\materia{Teoría de lenguajes}
\titulo{Recuperatorio Trabajo pr\'actico}
\subtitulo{Parser}
\hspace{6cm}
\integrante{Acosta, Javier Sebastian}{338/11}{acostajavier.ajs@gmail.com}
\integrante{Mastropasqua Nicolas Ezequiel}{828/13}{mastropasqua.nicolas@gmail.com}
\integrante{Negri, Franco}{893/13}{franconegri2004@hotmail.com}

\palabrasClave{TP}
  % Reconocimiento caras. PCA. Power Method. Deflation. Autovalores. Autovectores. Matriz
  % semi definida positiva.

\resumen{Este trabajo consiste en parser estilo c++}
% \resumen{El presente trabajo analiza los algor\'itmos de resoluci\'on de sistemas de ecuaciones %
% Gauss y LU mediante una simulaci\'on del c\'alculo de Isotermas en hornos industriales. \\ %
% Expone un sistema de ecuaciones para detectar la Isoterma de 500 grados y analiza el
% comportamiento de cada algoritmo en funci\'on a la discretizaci\'on de los puntos dentro del
% horno, cantidad de instancias a resolver y variaci\'on de instantes.\\ % Finalmente, detalla las
% conclusiones obtenidas mediante la comparaci\'on de ambos algoritmos en las circunstancias
% mencionadas.\\
% }
%\hypersetup{%
 % Para que el PDF se abra a página completa.
% pdfstartview= {FitH \hypercalcbp{\paperheight-\topmargin-1in-\headheight}},
% pdfauthor={Acosta, Gomez, Jabalera, Kodelia, Russo, Vuotto},
% pdfsubject={TP1}
%}

\parskip=5pt % 10pt es el tamaño de fuente

% Pongo en 0 la distancia extra entre ítemes.
\let\olditemize\itemize
\def\itemize{\olditemize\itemsep=0pt}

% Acomodo fancyhdr <- Creo que es el encabezado de pagina
\pagestyle{fancy}
\thispagestyle{fancy}
\addtolength{\headheight}{1pt}
%\lhead{Acosta, Negri}
%\rhead{1$^{do}$ Cuatrimestre 2016}
\cfoot{\thepage}
\renewcommand{\footrulewidth}{0.4pt}




%Pagina de titulo e indice
\thispagestyle{empty}

\maketitle


