\subsection{PCA}
EL primer modelo utilizado para intentar clasificar los datos será el de Analisis de Componentes Principales. Para ello utilizamremos dos algoritmos basados en aprendisaje Hebbiano y reduciremos las instancias de entrenamiento a 3 dimenciones. Lo que esperamos observar es que aquellas instancias que pertenecen a una misma clase de empresa se encuentran cercas unas de otras, pudiendo observar \"nuves\" de puntos bien definidas.

\subsubsection{Implementación}

En particular los algoritmos utilizados serán los de $Oja$ y $Sanger$. Teniendo una complegidad computacional identica y siendo los algoritmos muy similares, lo distintivo entre estos dos metodos es que $Sanger$ ordenará las componentes prinsipales de mayor a menor de acuerdo a sus autovalores mientras que $Oja$ no.

El pseodocodigo utilizado para aprendisaje del algoritmo $Oja$ será:

\begin{algorithm}
\begin{algorithmic}[1]\parskip=1mm
 \caption{ Algoritmo De Oja}
 \STATE{Para toda instancia de entrenamiento, x}
 \STATE{\quad $y = x.W$}
 \STATE{\quad $\tilde x = y.W^T$}
 \STATE{\quad $\Delta W = learning\_rate ((x - \tilde x)^T . y$}
\end{algorithmic}
\end{algorithm}

Mientras que el de $Sanger$

\begin{algorithm}
\begin{algorithmic}[1]\parskip=1mm
 \caption{ Algoritmo De Sanger}
 \STATE{U = Matriz Triangular Superior Con 1s}
 \STATE{Para toda instancia de entrenamiento, x}
 \STATE{\quad $y = x.W$}
 \STATE{\quad $\tilde x = W(y^T.U)$}
 \STATE{\quad $\Delta W = learning\_rate ((x^T - \tilde x) . y$}
\end{algorithmic}
\end{algorithm}

Utilizando el paquete numpy de python es posible traducir este codigo de manera casi exacta y de esa manera aprovechar las optimizaciones matriciales que se realizan sobre los datos.

\subsubsection{Experimentación}

Para la experimentación entrenamos la red con parte del set de datos que nos fue entregado

%Neural Networks Haykin
% In Chatterjee et al. (1998), the convergence properties of the GHA algorithm
% described in Eq. (8.91) are investigated. The analysis presented therein shows that
% increasing  leads to faster convergence and larger asymptotic mean-square error, which
% is intuitively satisfying. In that paper, the tradeoff between the accuracy of computation
% and speed of learning is made explicit.


\subsection{Mapeo de Características}

En este apartado construiremos un modelo de mapeo de caracteristicas auto-organizado con la intención de clasificar los documentos en un arreglo de dos dimenciones. Para ello utilizaremos el algoritmo de Kohonen sobre los datos de entrenamiento y una vez que la red haya convergido, graficaremos los las respuestas obtenidas en el plano.